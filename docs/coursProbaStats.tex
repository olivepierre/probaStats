\documentclass[]{book}
\usepackage{lmodern}
\usepackage{amssymb,amsmath}
\usepackage{ifxetex,ifluatex}
\usepackage{fixltx2e} % provides \textsubscript
\ifnum 0\ifxetex 1\fi\ifluatex 1\fi=0 % if pdftex
  \usepackage[T1]{fontenc}
  \usepackage[utf8]{inputenc}
\else % if luatex or xelatex
  \ifxetex
    \usepackage{mathspec}
  \else
    \usepackage{fontspec}
  \fi
  \defaultfontfeatures{Ligatures=TeX,Scale=MatchLowercase}
\fi
% use upquote if available, for straight quotes in verbatim environments
\IfFileExists{upquote.sty}{\usepackage{upquote}}{}
% use microtype if available
\IfFileExists{microtype.sty}{%
\usepackage{microtype}
\UseMicrotypeSet[protrusion]{basicmath} % disable protrusion for tt fonts
}{}
\usepackage[margin=1in]{geometry}
\usepackage{hyperref}
\hypersetup{unicode=true,
            pdftitle={Probabilités et statistiques},
            pdfauthor={Pierre-Damien Olive},
            pdfborder={0 0 0},
            breaklinks=true}
\urlstyle{same}  % don't use monospace font for urls
\usepackage{natbib}
\bibliographystyle{apalike}
\usepackage{longtable,booktabs}
\usepackage{graphicx,grffile}
\makeatletter
\def\maxwidth{\ifdim\Gin@nat@width>\linewidth\linewidth\else\Gin@nat@width\fi}
\def\maxheight{\ifdim\Gin@nat@height>\textheight\textheight\else\Gin@nat@height\fi}
\makeatother
% Scale images if necessary, so that they will not overflow the page
% margins by default, and it is still possible to overwrite the defaults
% using explicit options in \includegraphics[width, height, ...]{}
\setkeys{Gin}{width=\maxwidth,height=\maxheight,keepaspectratio}
\IfFileExists{parskip.sty}{%
\usepackage{parskip}
}{% else
\setlength{\parindent}{0pt}
\setlength{\parskip}{6pt plus 2pt minus 1pt}
}
\setlength{\emergencystretch}{3em}  % prevent overfull lines
\providecommand{\tightlist}{%
  \setlength{\itemsep}{0pt}\setlength{\parskip}{0pt}}
\setcounter{secnumdepth}{5}
% Redefines (sub)paragraphs to behave more like sections
\ifx\paragraph\undefined\else
\let\oldparagraph\paragraph
\renewcommand{\paragraph}[1]{\oldparagraph{#1}\mbox{}}
\fi
\ifx\subparagraph\undefined\else
\let\oldsubparagraph\subparagraph
\renewcommand{\subparagraph}[1]{\oldsubparagraph{#1}\mbox{}}
\fi

%%% Use protect on footnotes to avoid problems with footnotes in titles
\let\rmarkdownfootnote\footnote%
\def\footnote{\protect\rmarkdownfootnote}

%%% Change title format to be more compact
\usepackage{titling}

% Create subtitle command for use in maketitle
\providecommand{\subtitle}[1]{
  \posttitle{
    \begin{center}\large#1\end{center}
    }
}

\setlength{\droptitle}{-2em}

  \title{Probabilités et statistiques}
    \pretitle{\vspace{\droptitle}\centering\huge}
  \posttitle{\par}
    \author{Pierre-Damien Olive}
    \preauthor{\centering\large\emph}
  \postauthor{\par}
      \predate{\centering\large\emph}
  \postdate{\par}
    \date{Mis à jour le 2019-03-14}

\usepackage{booktabs}

\usepackage{amsthm}
\newtheorem{theorem}{Théorème}[chapter]
\newtheorem{lemma}{Lemme}[chapter]
\newtheorem{corollary}{Corollaire}[chapter]
\newtheorem{proposition}{Proposition}[chapter]
\newtheorem{conjecture}{Conjecture}[chapter]
\theoremstyle{definition}
\newtheorem{definition}{Définition}[chapter]
\theoremstyle{definition}
\newtheorem{example}{Exemple}[chapter]
\theoremstyle{definition}
\newtheorem{exercise}{Exercice}[chapter]
\theoremstyle{remark}
\newtheorem*{remark}{Remarque}
\newtheorem*{solution}{Solution}
\let\BeginKnitrBlock\begin \let\EndKnitrBlock\end
\begin{document}
\maketitle

{
\setcounter{tocdepth}{1}
\tableofcontents
}
\chapter{Introduction}\label{introduction}

Markdown permet plein de mises en forme, en \textbf{gras} ou en
\emph{italique}. On peut également faire des formules (\(E = mc^2\)).

\part{Probabilités}\label{part-probabilites}

\chapter{Rappels de probabilités}\label{rappels-de-probabilites}

On fait des probabilités avec \(\mathbb{P}\).

\section{début des rappels}\label{debut-des-rappels}

\section{fin des rappels}\label{fin-des-rappels}

\subsection{rappels 1}\label{rappels-1}

\subsection{rappels 2}\label{rappels-2}

\begin{itemize}
\tightlist
\item
  r1
\item
  r2

  \begin{itemize}
  \tightlist
  \item
    r21
  \item
    r22
  \end{itemize}
\end{itemize}

\begin{enumerate}
\def\labelenumi{\arabic{enumi}.}
\tightlist
\item
  A
\item
  B
\item
  C
\end{enumerate}

\chapter{tests avec des matrices}\label{tests-avec-des-matrices}

\[X = \begin{bmatrix}1 & x_{1}\\
1 & x_{2}\\
1 & x_{3}
\end{bmatrix}\]

\begin{equation} 
  f\left(k\right) = \binom{n}{k} p^k\left(1-p\right)^{n-k}
  \label{eq:binom}
\end{equation}

C'est l'équation \eqref{eq:binom}

\BeginKnitrBlock{theorem}
\protect\hypertarget{thm:unnamed-chunk-1}{}{\label{thm:unnamed-chunk-1} }Je
fais mon théorème
\EndKnitrBlock{theorem}

\part{Statistiques}\label{part-statistiques}

\chapter{Modèle linéaire}\label{modele-lineaire}

\appendix


\chapter{Transformée de Laplace}\label{transformee-de-laplace}

\bibliography{book.bib,packages.bib}


\end{document}
